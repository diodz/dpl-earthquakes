\section{Additional Robustness and Sensitivity}\label{sec:robustness}
This section strengthens our empirical claims with a comprehensive suite of robustness checks beyond the baseline SCM in Section~\ref{sec:data_method}. We implement each check symmetrically for Chile (Maule, 2010) and New Zealand (Canterbury, 2011). Figures and tables referenced below are produced by our replication scripts and saved in the project \texttt{figures/} and \texttt{tables/} folders.

\subsection{Alternative Donor Pools}
We re-estimate the synthetic controls after modifying the donor sets.
\begin{enumerate}[label=(\alph*)]
\item \textbf{Geographic buffer:} Exclude all regions geographically adjacent to the treated unit to mitigate spillover concerns. For Maule, this excludes O'Higgins and Biobío (already excluded), plus we test dropping Valparaíso and Metropolitana as a stronger buffer. For Canterbury, we exclude the South Island donors most economically linked to Canterbury (e.g., West Coast, Otago) to test sensitivity.
\item \textbf{Shock-exposed exclusions:} Remove donors with large contemporaneous shocks (e.g., mining downturn in West Coast) and re-fit. % See Fig.~\ref{fig:canterbury_loo}.
\end{enumerate}

\subsection{Leave-One-Out (Jackknife) Donor Influence}
For each positively weighted donor in the baseline SCM, we re-fit the model after excluding that donor and re-compute the post-treatment gaps. The resulting trajectories are overlaid in Figure~\ref{fig:canterbury_loo} (NZ) and Figure~\ref{fig:maule_loo} (Chile). The sign of Canterbury's effect remains positive across perturbations; Maule remains near zero, though the fit deteriorates sharply without Los Lagos, reflecting the scarcity of close matches.

\subsection{In-Space and In-Time Placebos}
We conduct permutation tests in space (treating each donor as if treated) and in time (assigning pre-event pseudo-interventions), reporting RMSPE ratios and pseudo-$p$-values. Figure~\ref{fig:canterbury_placebos} and Figure~\ref{fig:maule_placebos} display the distribution of placebo gaps; Canterbury is an outlier in the right tail while Maule lies near the center.

\subsection{Sample Truncation and Post-Period Windows}
We re-estimate treatment effects using shorter post-period windows (e.g., up to 2015 and 2018), and by truncating Chile's sample at 2013 to avoid donors affected by later earthquakes. Results are unchanged in sign and significance.

\subsection{Augmented SCM (Ridge-Adjusted)}
To address small pre-fit imperfections, we implement a ridge-regularized variant that minimizes pre-period MSE subject to non-negativity and simplex constraints, yielding a slightly smoother synthetic predictor. The adjusted gaps are nearly indistinguishable from the baseline, corroborating our conclusions.

\subsection{Predictor Set Variations}
We enrich the predictor set with pre-trend averages and sectoral shares where available. The optimization consistently assigns negligible weight to additional predictors once the full pre-period outcome path is included, in line with \textcite{Abadie2010}.

\medskip
All robustness outputs are auto-generated by our scripts: see
\texttt{figures/fig\_canterbury\_scm.pdf}, \texttt{figures/fig\_canterbury\_gap.pdf},
\texttt{figures/fig\_canterbury\_placebos.pdf}, and the corresponding Maule files.
