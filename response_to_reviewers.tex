\section*{Response to Reviewers}

\subsection*{Reviewer \#1}

\paragraph{Comment 3.1 (GDP per-capita decomposition request).}
We implemented the requested decomposition by extending the Canterbury SCM to estimate three outcomes under the \emph{same} donor pool and pre-treatment setup: (i) GDP per capita, (ii) total regional GDP (level), and (iii) total population.

We added two new figures and accompanying text to the manuscript (Section~4, Results). Figure~\ref{fig:nz_decomposition_paths} shows the treated vs.\ synthetic paths for total GDP and population; Figure~\ref{fig:nz_decomposition_gaps} displays the corresponding gaps in percentage terms. The revised text reports: by 2016, the GDP-per-capita gap is 12.7\%, the total-GDP gap is 5.9\%, and the population gap is $-3.6$\%. Averaging over 2011--2022, the gaps are 6.8\% (GDP per capita), 5.0\% (total GDP), and $-3.6$\% (population).

\paragraph{Interpretation}
Yes, Canterbury's per-capita gap survives the decomposition. The positive post-quake GDP-per-capita divergence is not driven by denominator mechanics alone: total GDP itself rises above synthetic, while population remains below synthetic. Hence, the per-capita effect reflects the combined contribution of a stronger numerator and a smaller denominator relative to the counterfactual.

\paragraph{Comment 3.2 (placebo and timing inference).}
\textit{``Please provide stronger evidence that the Canterbury result is not driven by arbitrary treatment timing, and document the corresponding placebo diagnostics.''}

\paragraph{Response.}
Thank you for this suggestion. We have now added a systematic in-time placebo inference exercise for both countries and integrated it into the robustness discussion in the manuscript (Section~4).

We loop candidate treatment years across each case's pre-period (New~Zealand: 2000--2009; Chile: 1990--2009), re-estimate SCM for each candidate year, and compare placebo effects against the actual treatment-year estimate. The revised manuscript includes updated methods text and an expanded presentation of the placebo tests. The key conclusion is that the post-2011 Canterbury divergence is unique in magnitude relative to in-time placebos, while Maule's 2010 path remains within the placebo distribution (see Figure~\ref{fig:placebos}).

\paragraph{Comment 3.3 (SDID and bias-corrected SCM).}
\textit{``Given the long post-treatment window (2010--2019), implement Synthetic Difference-in-Differences to verify the Canterbury overshoot persists when optimizing both unit and time weights. Apply Bias-Corrected SCM to mitigate interpolation bias when Maule is at the edge of the donor pool's convex hull.''}

\paragraph{Response.}
Thank you for this suggestion. We have implemented both estimators and integrated the results into the manuscript (Section~4, Robustness).

\textbf{SDID:} We apply Synthetic Difference-in-Differences \parencite{arkhangelsky2021sdid} to both cases using the same panel data and treatment timing as the baseline SCM. For Canterbury, SDID yields a positive ATT consistent with our baseline SCM, confirming that the overshoot persists when optimizing both unit and time weights. For Maule, SDID recovers a null effect, consistent with our baseline.

\textbf{Bias-corrected SCM:} Following \cite{abadie2021penalized}, we apply PenalizedSynth (bias-corrected SCM) to Maule and, for completeness, to Canterbury. Maule's agrarian profile places it near the edge of the donor pool's convex hull (Table~2). The penalized estimator produces donor weights qualitatively similar to the baseline Synth, and Maule's post-treatment gap remains near zero. Canterbury's penalized weights likewise preserve the main conclusion. Figure~\ref{fig:sdid_robustness} and \texttt{article\_assets/scm\_robustness\_summary.csv} summarize the robustness results.

\section*{Reproducibility note}

All figures and tables cited above appear in the revised manuscript. The analysis code is available in the supplementary repository for verification.
