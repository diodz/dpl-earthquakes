\section*{Response to Reviewers}

\subsection*{Reviewer \#1}

\paragraph{Comment 3.1 (GDP per-capita decomposition request).}
We implemented the requested decomposition by extending the Canterbury SCM pipeline to estimate three outcomes under the \emph{same} donor pool and pre-treatment setup: (i) GDP per capita, (ii) total regional GDP (level), and (iii) total population.

\paragraph{Exact outputs generated}
All outputs are now written to \texttt{article\_assets/}:
\begin{itemize}
    \item \texttt{nz\_population\_paths.png}
    \item \texttt{nz\_population\_gap.png}
    \item \texttt{nz\_total\_gdp\_paths.png}
    \item \texttt{nz\_total\_gdp\_gap.png}
    \item \texttt{nz\_outcome\_extension\_tables.xlsx}
    \item \texttt{nz\_outcome\_extension\_summary.csv}
\end{itemize}

The summary file reports:
\begin{itemize}
    \item GDP per capita gap: 12.717\% in 2016; post-2011 average gap of 6.825\%.
    \item Total GDP gap: 5.918\% in 2016; post-2011 average gap of 5.011\%.
    \item Population gap: $-3.631$\% in 2016; post-2011 average gap of $-3.628$\%.
\end{itemize}

\paragraph{Interpretation}
Yes, Canterbury's per-capita gap survives the decomposition. The positive post-quake GDP-per-capita divergence is not driven by denominator mechanics alone: total GDP itself rises above synthetic, while population remains below synthetic. Hence, the per-capita effect reflects the combined contribution of a stronger numerator and a smaller denominator relative to the counterfactual.

\paragraph{Comment 3.2 (placebo and timing inference).}
\textit{``Please provide stronger evidence that the Canterbury result is not driven by arbitrary treatment timing, and document the corresponding placebo diagnostics.''}

\paragraph{Response.}
Thank you for this suggestion. We have now added a systematic in-time placebo inference exercise for both countries and integrated it into the robustness discussion in the manuscript.

Specifically, we now loop candidate treatment years across each case's pre-period (New~Zealand: 2000--2009; Chile: 1990--2009), re-estimate SCM for each candidate year, and store post-placebo gaps and post/pre RMSPE ratios. We export the results as tidy CSV artifacts and generate compact distributional figures that compare placebo effects against the actual treatment-year estimate.

\begin{itemize}
  \item New~Zealand tidy summary: \texttt{article\_assets/nz\_in\_time\_placebo\_summary.csv}
  \item New~Zealand compact figure: \texttt{article\_assets/nz\_in\_time\_placebo\_compact.png}
  \item Chile tidy summary: \texttt{article\_assets/chile\_in\_time\_placebo\_summary.csv}
  \item Chile compact figure: \texttt{article\_assets/chile\_in\_time\_placebo\_compact.png}
\end{itemize}

We also added corresponding methods and results text in \texttt{main.tex}. The key conclusion from this added exercise is that the post-2011 Canterbury divergence is unique in magnitude relative to in-time placebos, while Maule's 2010 path remains within the placebo distribution.

\section*{Reproducibility note}

All updated figures and analysis outputs can be regenerated from the repository root with one command:

\begin{verbatim}
python run_analysis.py
\end{verbatim}

This command executes the analysis pipeline and writes outputs to \texttt{article\_assets/}, which is the canonical artifact path used by \texttt{main.tex}.
