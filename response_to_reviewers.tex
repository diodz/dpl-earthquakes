\section*{Response to Reviewers}

\subsection*{Reviewer \#1}

\paragraph{Comment 2.1 (wealth vs.\ flow; broken-window concern).}
\textit{``The Canterbury overshoot could reflect a replacement or rebuilding boom rather than a genuine productivity gain. Please clarify whether the estimated effects capture higher wealth or merely higher flows, and address the classic broken-window critique more directly.''}

\paragraph{Response (placeholder).}
We acknowledge this important conceptual distinction between wealth losses and flow-based reconstruction dynamics. In the current revision, we flag the issue in the discussion but do not yet add a full additional empirical decomposition beyond the robustness checks already reported. A more complete treatment of wealth-versus-flow mechanisms is planned for a subsequent revision.

\paragraph{Comment 2.2 (cross-case comparability and spatial heterogeneity).}
\textit{``The Maule and Canterbury shocks differ in magnitude and geography (dispersed/rural vs urban hub). Please control for spatial damage intensity before attributing differences to institutions.''}

\paragraph{Response.}
Implemented with three additions in the revised manuscript:
\begin{enumerate}
  \item \textbf{Damage-normalized comparability checks.} In the robustness section we now report descriptive scaling metrics using common damage denominators from Table~\ref{tab:damage_compare}: peak SCM gap per US\$1 billion of direct damage and peak SCM gap per 1 percentage point of GDP-loss share. Canterbury remains positive under both normalizations, while Maule remains near zero.
  \item \textbf{Symmetric timing windows across cases.} We now report treatment-year sensitivity for both countries (Chile: 2009/2010/2011 starts; New~Zealand: 2010/2011/2012 starts) to avoid attributing differences to arbitrary timing definitions.
  \item \textbf{Spatial damage controls in measurement.} We added urban-core night-lights checks for both countries and Christchurch-centered 30/50/80 km buffer tests for Canterbury, in addition to the geographically progressive donor-exclusion tests in Appendix~\ref{sec:spillover_appendix}.
\end{enumerate}

Interpretation: these controls do not force physical equivalence between a subduction mega-event and an urban sequence; rather, they verify that the qualitative economic contrast (Canterbury positive overshoot, Maule near-null) is robust after explicitly accounting for differences in damage scale, timing conventions, and within-region spatial concentration.

\paragraph{Comment 2.3 (insurance as exogenous external inflow).}
\textit{``Insurance payouts may operate as large, partly exogenous external inflows. Please clarify the role of insurance in the Canterbury case and whether it should be treated as an outside shock rather than an institutional channel.''}

\paragraph{Response (placeholder).}
The current revision discusses insurance qualitatively as part of the institutional environment but does not yet implement new empirical tests that separately identify insurance flows as an external inflow channel. We leave a fuller treatment of insurance mechanisms and their empirical separation from other institutional responses to a future robustness extension.

\paragraph{Comment 3.1 (GDP per-capita decomposition request).}
We implemented the requested decomposition as a general interpretability requirement for Y/L in \emph{both} treated regions, subject to data availability. For Canterbury we extend the SCM to estimate three outcomes under the \emph{same} donor pool and pre-treatment setup: (i) GDP per capita, (ii) total regional GDP (level), and (iii) total population. For Maule, the Chilean regional source does not provide a population/total-GDP series covering the full SCM window, so we report the decomposition evidence available for the observed period and state this limitation explicitly.

We added decomposition figures and text for both regions (Section~4, Results). For Canterbury, Figure~\ref{fig:nz_decomposition_paths} shows the treated vs.\ synthetic paths for total GDP and population; Figure~\ref{fig:nz_decomposition_gaps} displays the gaps: by 2016, the GDP-per-capita gap is 12.7\%, the total-GDP gap is 5.9\%, and the population gap is $-3.6$\%; averaging over 2011--2019, the gaps are 7.8\% (GDP per capita), 6.5\% (total GDP), and $-3.6$\% (population). For Maule, regional population (and thus total GDP) are not available over the full SCM window in our Chilean data source, so we report the GDP-per-capita decomposition only: Figure~\ref{fig:maule_decomposition_paths} and Figure~\ref{fig:maule_decomposition_gaps} show that Maule's GDP-per-capita gap remains near zero, and we flag this as partial evidence given the data constraint.

\paragraph{Interpretation}
Canterbury's per-capita gap survives the decomposition: total GDP rises above synthetic and population remains below synthetic, so the per-capita effect reflects both a stronger numerator and a smaller denominator. For Maule, the GDP-per-capita decomposition shows null-like gaps, confirming that the absence of a per-capita overshoot is not driven by composition effects in the available data.

\paragraph{Comment 3.2 (uniform confidence sets and timing sensitivity).}
\textit{``Please move beyond placebo spaghetti plots and provide uniform confidence sets over the post-treatment path, including timing/fitting sensitivity checks.''}

\paragraph{Response.}
We address both parts of the comment as follows.

\textbf{Uniform confidence sets and timing sensitivity:} We expanded the timing analysis to provide full treatment-year sensitivity outputs. For New~Zealand we report 2010-start, 2011-start (preferred), and 2012 sequence-aware start; for Chile, 2010 baseline plus nearby placebo starts (2009 and 2011). For each timing choice we report pre/post fit quality, post/pre RMSPE ratio, and placebo ranking (Table~\ref{tab:timing_sensitivity}, Figures~\ref{fig:timing_sensitivity_paths}--\ref{fig:timing_sensitivity_rmspe}). We also report placebo pre-fit filters for Chile (2$\times$, 5$\times$, 10$\times$ treated pre-RMSPE). Uniform confidence sets over the post-treatment path are reported in Figure~\ref{fig:uniform_sets}. Main conclusions are stable: Maule's uniform set includes zero throughout the post period, while Canterbury shows non-zero positive years concentrated in early-to-mid rebuild years, with later attenuation.

\textbf{Rolling in-time placebo (every pre-treatment year):} The referee requested a rolling in-time placebo for \emph{every} possible year in the pre-treatment period (e.g., 2000--2009). We now implement this explicitly: we re-estimate SCM for each candidate treatment year $T$ in the pre-treatment window and plot the resulting gap path. We require at least three pre-treatment years for each fit, so the candidate set is New~Zealand: $T \in \{2003,\ldots,2009, 2011\}$ (years 2000--2002 omitted) and Chile: $T \in \{1993,\ldots,2009, 2010\}$ (years 1990--1992 omitted). The resulting placebo paths are reported in Figure~\ref{fig:rolling_in_time_placebo}: each line's placebo treatment year is indicated by color (colorbar); the thick red line is the path for the actual treatment year. The figure shows that the post-treatment divergence is unique to the actual treatment year (Canterbury's gap rises only when 2011 is used; Maule remains near zero for 2010 and for all placebo years). This completes the requested rolling in-time placebo exercise.

\paragraph{Comment 3.3 (SDID and bias-corrected SCM robustness).}
\textit{``Given the long post-treatment window, please implement SDID and a bias-corrected/penalized SCM (especially for Maule) to test robustness to inexact matching.''}

\paragraph{Response.}
Implemented. We added robustness analyses that estimate:
\begin{enumerate}
  \item SDID for both Canterbury and Maule, and
  \item penalized (bias-corrected) SCM following the Abadie--L'Hour framework (reported for both countries; requested case Maule included).
\end{enumerate}

The revised manuscript reports these estimates in a new robustness table and figure (Table~\ref{tab:sdid_penalized}, Figure~\ref{fig:sdid_penalized_gaps}). Numerically:
\begin{itemize}
  \item \textbf{Canterbury}: baseline SCM average post-gap (2011--2019) is \(+7.8\%\); SDID is \(+7.8\%\); penalized SCM is \(+6.5\%\).
  \item \textbf{Maule}: baseline SCM average post-gap (2010--2019) is \(-1.7\%\); SDID is \(-5.1\%\); penalized SCM is \(-1.5\%\).
\end{itemize}

Hence, Canterbury's overshoot persists when optimizing both unit and time weights (SDID), while Maule remains non-positive under all alternative estimators. This directly addresses concerns about long-window robustness and edge-of-convex-hull matching for Maule.

\paragraph{Comment 3.6 (circularity and crowding-out in sectoral interpretation).}
\textit{``Sectoral interpretation appears circular; please quantify crowding-out and provide parallel Chile/New~Zealand sectoral inference.''}

\paragraph{Response.}
Implemented. We added sectoral analyses and integrated the results into the manuscript text and appendix.

\begin{itemize}
  \item We now include \textbf{parallel sectoral SCM figures for both countries}, not only Canterbury (new Maule figure added).
  \item We corrected quantitative wording: Canterbury construction rises from roughly \(6.1\%\) to \(10.0\%\), i.e., about \(+63\%\), \textbf{not} a doubling.
  \item We report placebo-based inference for sectoral outcomes in a new summary table.
  \item We quantify crowding-out in both \textbf{share} and \textbf{absolute level} terms.
\end{itemize}

Main quantitative additions (post-treatment averages):
\begin{itemize}
  \item \textbf{Canterbury}: construction share gap \(+2.53\) p.p. (pseudo-\(p<0.01\)); non-construction share gap \(-2.13\) p.p. (pseudo-\(p<0.01\)); non-construction \emph{level} gap \(+911.23\) (\(+3.53\%\) of synthetic; pseudo-\(p=0.87\)).
  \item \textbf{Maule}: construction share gap \(+1.82\) p.p. (pseudo-\(p=0.33\)); non-construction share gap \(-2.01\) p.p. (pseudo-\(p=0.33\)); non-construction level gap \(-12.51\) (\(-33.98\%\) of synthetic; pseudo-\(p=0.42\)).
\end{itemize}

Interpretation: Canterbury exhibits a strong \emph{share} reallocation toward construction, but no robust absolute contraction in non-construction output (consistent with temporary net stimulus plus reallocation). Maule does not exhibit a statistically robust sectoral overshoot.

We also added \textbf{alternative grouping sensitivity} for both countries (goods vs services non-construction aggregates), reported in the appendix table.


\paragraph{Comment 3.4 (SUTVA/spillover diagnostics).}
\textit{``Current donor-drop checks are useful but do not directly test economic spillover channels (migration, labor reallocation, trade/input-output links). Please strengthen SUTVA diagnostics with explicit flow evidence and geographically structured donor exclusions.''}

\paragraph{Response.}
Implemented. We added a full appendix section (Appendix~\ref{sec:spillover_appendix}) that addresses this concern through three complementary diagnostics:

\begin{enumerate}
  \item \textbf{Explicit spillover proxy evidence:} We assemble population-flow indicators, construction-sector shift diagnostics (labour reallocation proxy), and inter-regional sector-structure correlations (economic linkage index) for both Chile and New~Zealand (Tables~\ref{tab:pop_flow_diagnostics}--\ref{tab:economic_linkage}).
  \item \textbf{Geographically structured progressive donor exclusions:} We define concentric geographic ``rings'' around each treated region and re-estimate SCM under progressively restrictive donor pools (4 rings for Chile, 5 rings for New~Zealand). Results are reported in Table~\ref{tab:spillover_exclusion_summary} and Figure~\ref{fig:spillover_gap_paths}.
  \item \textbf{Side-by-side baseline vs.\ spillover-robust estimates:} Figure~\ref{fig:spillover_sensitivity_bar} provides a direct comparison of mean post-treatment gaps under each exclusion ring.
\end{enumerate}

Main findings:
\begin{itemize}
  \item \textbf{Canterbury:} The positive effect is qualitatively robust to \emph{all} geographic exclusion specifications. The mean post-gap ranges from $+5.4\%$ (Ring~1, excluding West Coast) to $+12.7\%$ (Ring~4, most restrictive). The sign never reverses. Population and construction-sector diagnostics show no evidence of spillovers to neighbouring donor regions.
  \item \textbf{Maule:} The null/non-positive result holds under Rings 1--2 (excluding O'Higgins, Araucanía, and RM Santiago). Under the most restrictive Ring~3 (only 6 distant donors), the sign reverses but pre-treatment fit deteriorates, indicating convex-hull breakdown rather than a genuine positive effect. Population-flow diagnostics confirm no differential migration to adjacent regions.
\end{itemize}

Sensitivity statement: Canterbury's positive finding survives all reasonable SUTVA-motivated exclusions. Maule's null result is robust under plausible exclusion rings (Rings 1--2) but becomes unreliable when the donor pool is too severely restricted.

\paragraph{Comment 3.5 (social capital/looting narrative).}
\textit{``The discussion invokes social capital and looting narratives, but this mechanism is not well-supported by direct evidence. Please either strengthen the evidence base or temper the claims.''}

\paragraph{Response (placeholder).}
We agree that the narrative mechanisms around social capital and looting require stronger direct evidence. In the current revision we retain a cautious, probabilistic interpretation and acknowledge this limitation, but we do not yet add dedicated new data or tests; a more systematic treatment of this mechanism is left for future work.

\paragraph{Comment 3.7 (insurance-liquidity channel).}
\textit{``If insurance is central to the proposed mechanism, please provide more direct tests of an insurance-liquidity channel rather than inferring it indirectly.''}

\paragraph{Response (placeholder).}
At present we do not introduce new empirical tests that separately trace an insurance-liquidity channel (e.g., via payout timing or coverage heterogeneity). We instead interpret insurance as one component of a broader institutional package and note that isolating its causal contribution remains an open task for future extensions.

\paragraph{Minor comment (institutional metrics are national-level only).}
\textit{``Institutional quality indicators are reported only at the national level. Please discuss whether subnational institutional heterogeneity could affect the interpretation.''}

\paragraph{Response (placeholder).}
We acknowledge that our institutional indicators are measured at the national level and that subnational heterogeneity is not directly captured. The current revision emphasizes this limitation qualitatively, but we do not yet add new subnational institutional measures; this remains a potential avenue for future data collection.

\paragraph{Minor comment (mechanism circularity; add GFCF/consumption outcomes).}
\textit{``The mechanism discussion risks circularity. Please add complementary outcomes (e.g., gross fixed capital formation or consumption) to clarify whether the overshoot is driven by investment or broader welfare gains.''}

\paragraph{Response (placeholder).}
In this revision we keep GDP-per-capita as the primary outcome and focus on robustness within that domain. We agree that additional macro aggregates such as GFCF or consumption would improve mechanism identification, but those extensions are not yet implemented in the current empirical package.

\paragraph{Minor comment (data-quality corroboration via night-time lights).}
\textit{``Corroborate regional GDP findings with an objective proxy such as NTL intensity.''}

\paragraph{Response.}
Implemented. We added analyses that build annual regional NTL series and run parallel SCM validation for Chile (Maule) and New Zealand (Canterbury). The procedure: (i) aggregates a harmonized DMSP-OLS/VIIRS annual product to regional boundaries used in our SCM design; (ii) re-estimates SCM on NTL outcomes for both treated regions (same treatment timing logic); and (iii) reports product-sensitivity and spatial-sensitivity checks (urban-core mask in Maule; Christchurch buffers for Canterbury). Results are reported in the revised manuscript (Figures~\ref{fig:ntl_validation} and \ref{fig:ntl_spatial_sensitivity} and related text).

\paragraph{Interpretation}
For Canterbury, NTL confirms the GDP-based result: positive post-treatment divergence (about +10.0\% in 2016; average post-gap about +7.8\%). For Maule, NTL diverges from the GDP-null result (positive NTL post-gap). We now state this explicitly in the manuscript and discuss it as evidence that remote-sensing proxies can capture broader spatial re-lighting and infrastructure restoration dynamics that do not map one-to-one to measured value-added in regional GDP accounts. Product and spatial sensitivity checks preserve this qualitative pattern.

\subsection*{Reviewer \#2}

\paragraph{Major Comment 1 (SUTVA concerns: O'Higgins/Auckland contamination).}
\textit{``SUTVA may be violated if adjacent or economically linked donor regions are contaminated by earthquake spillovers (e.g., O'Higgins for Chile; Auckland for New~Zealand).''}

\paragraph{Response.}
We address this concern comprehensively in the new SUTVA/spillover diagnostics appendix (Appendix~\ref{sec:spillover_appendix}) and in the response to Reviewer~\#1 Comment~3.4 above. Specifically:

\begin{itemize}
  \item For O'Higgins (Chile): Excluding O'Higgins from the donor pool (Ring~1) changes the Maule mean gap from $-4.5\%$ to $-5.8\%$---the null conclusion is unchanged. Population-flow and construction-sector diagnostics show no evidence that O'Higgins was differentially affected relative to more distant donors. The economic linkage index places O'Higgins at a moderate correlation ($0.38$) with Maule, lower than Araucanía ($0.74$) and Biobío ($0.70$, already excluded).
  \item For Auckland (New~Zealand): Excluding Auckland (Ring~4, the most restrictive exclusion) \emph{increases} the Canterbury gap from $+7.8\%$ to $+12.7\%$, suggesting that including Auckland as a donor---if anything---understates the Canterbury effect. This is consistent with Auckland potentially receiving positive spillovers (redirected government spending, temporary migration), which would bias the synthetic counterfactual upward and the treatment gap downward. Population diagnostics confirm that non-adjacent regions (including Auckland) experienced stronger post-quake population growth than adjacent regions.
\end{itemize}

We conclude that the specific O'Higgins and Auckland contamination concerns do not threaten either the Maule null finding or the Canterbury positive finding.

\paragraph{Major Comment 2 (mechanism identification remains conjectural).}
\textit{``The proposed mechanisms remain somewhat conjectural relative to the empirical evidence. Please strengthen the linkage between estimated effects and specific institutional or behavioral channels.''}

\paragraph{Response (placeholder).}
We recognize that fully pinning down mechanisms requires additional targeted evidence beyond what is feasible in the current revision. The manuscript now presents the institutional interpretation as conditional and probabilistic, but we do not yet add the richer micro or administrative data that would be needed for sharper mechanism tests.

\paragraph{Major Comment 5 (alternative confounders: commodity prices, exchange rates, macro cycle, and Region I).}
\textit{``Please address whether differential commodity prices, exchange-rate movements, post-GFC macro conditions, and other Chilean shocks (e.g., 2014 Iquique) could explain the estimated gaps.''}

\paragraph{Response.}
We have now addressed these confounders explicitly in the revised manuscript's robustness discussion.

\begin{itemize}
  \item \textbf{Exchange rates:} SCM is estimated separately within each country using constant-price regional series, so cross-country nominal exchange-rate movements cannot mechanically generate treated-vs.-synthetic gaps.
  \item \textbf{Commodity and macro-cycle channels:} the design compares each treated region to weighted domestic donors subject to the same national monetary/fiscal regime and broad external cycle, which differences out common macro shocks by construction.
  \item \textbf{Region I (Iquique 2014):} we implemented donor-pool exclusions for regions exposed to later major shocks; Maule's near-null result is unchanged when Region~I is excluded.
  \item \textbf{Canterbury and post-GFC recovery:} we now state more explicitly that reconstruction overlapped with the broader post-GFC upswing and a multi-shock sequence (2010--2011), which may have amplified short-run multipliers.
\end{itemize}

Accordingly, we frame institutions as an important mediator rather than the sole mechanism, and we retain the core result as a robust empirical asymmetry: Maule remains near-null while Canterbury shows a temporary positive overshoot that attenuates later in the sample.

\paragraph{Major Comment 3 (event comparability: magnitude, multi-shock sequence, urban/rural composition).}
\textit{``Please address whether physical and spatial differences between events, rather than institutions, explain divergent outcomes.''}

\paragraph{Response.}
Addressed in the revised introduction and robustness sections. We now: (i) explicitly frame comparability as a testable design problem, not an assumption; (ii) add damage-normalized SCM effect benchmarks based on common reported damage metrics; (iii) report symmetric timing-placebo windows across both countries; and (iv) include spatially focused validation checks (urban-core masks and Christchurch-centered buffers) to limit compositional bias from large heterogeneous regions. The results continue to show a robust Canterbury overshoot and a Maule near-null aggregate response. We therefore present the institutional interpretation as conditional and probabilistic (``institutions likely mediated outcomes''), while acknowledging remaining residual heterogeneity in shock physics and geography.

\paragraph{Major Comment 4 (sectoral analysis presentation and interpretation).}
\textit{``The sectoral SCM analysis is incomplete and asymmetrically presented. Please show Maule sectoral figures, quantify the construction-share increase more precisely, and report placebo-based sectoral inference.''}

\paragraph{Response.}
Implemented. As detailed in the response to Reviewer~\#1 Comment~3.6, we (i) add parallel Maule sectoral SCM figures alongside Canterbury, (ii) correct the wording on Canterbury's construction share (from ``nearly doubled'' to an increase from roughly 6.1\% to 10.0\%, about +63\%), and (iii) report placebo-based sectoral inference and crowding-out diagnostics in both share and level terms. These additions address the concerns about completeness, symmetry, and statistical support for the sectoral interpretation.

\paragraph{Minor Comment 1 (predictor selection and asymmetry).}
\textit{``Table 2 shows predictor means but does not explain how predictor weights were chosen or why some predictors (e.g., Fishing) appear only for Chile. Please clarify the predictor-weight selection procedure and justify cross-country asymmetries.''}

\paragraph{Response.}
Addressed. Section~3 now clarifies that we use the standard data-driven SCM predictor-weight optimization (Abadie et al., 2010) with a common baseline solver, and the robustness section reports sensitivity to alternative initialization/methods and to a harmonized reduced predictor set. We also explain that certain sectors (such as Fishing) are specific to Chile's published regional accounts and therefore have no direct analogue in New~Zealand, and we show that harmonizing to broader sectoral groupings does not materially change the substantive conclusions.

\paragraph{Minor Comment 2 (treatment timing clarification).}
\textit{``The treatment timing for Canterbury (2011 vs.\ 2010) needs clearer justification and explicit sensitivity analysis.''}

\paragraph{Response.}
Addressed. The revised manuscript makes the timing convention explicit (2011 as the preferred post-treatment start, reflecting the February 2011 event) and reports both 2010-start and 2012-start sensitivity specifications, with full placebo and timing diagnostics (Section~4.3 and timing-sensitivity table/figures), as well as the rolling in-time placebo analysis. These additions show that the qualitative Canterbury overshoot is not an artifact of the specific treatment-year choice.

\paragraph{Minor Comment 3 (welfare caveat prominence).}
\textit{``Please give greater prominence to welfare caveats, making clear that positive GDP gaps do not necessarily translate into unambiguous welfare gains.''}

\paragraph{Response (placeholder).}
We agree that GDP-based gaps are an imperfect proxy for welfare. The current revision strengthens textual caveats in the introduction and conclusion, but we do not yet introduce new welfare measures or distributional evidence; further work along these lines remains for a future iteration.

\paragraph{Minor Comment 4 (Table integration/formalization).}
\textit{``Please better integrate and formalize the interpretive table (e.g., current Table 4), clarifying its role in the argument and aligning it more tightly with the empirical results.''}

\paragraph{Response (placeholder).}
The present revision keeps the interpretive table but does not yet implement a full redesign or tighter formal linkage to the empirical sections. We intend to revisit this table's structure and integration in a subsequent round once additional robustness work is completed.

\section*{Reproducibility note}

All figures and tables cited above appear in the revised manuscript.
