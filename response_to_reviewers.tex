\section*{Response to Reviewers}

\subsection*{Reviewer \#1}

\paragraph{Comment 3.1 (GDP per-capita decomposition request).}
We implemented the requested decomposition by extending the Canterbury SCM to estimate three outcomes under the \emph{same} donor pool and pre-treatment setup: (i) GDP per capita, (ii) total regional GDP (level), and (iii) total population.

We added two new figures and accompanying text to the manuscript (Section~4, Results). Figure~\ref{fig:nz_decomposition_paths} shows the treated vs.\ synthetic paths for total GDP and population; Figure~\ref{fig:nz_decomposition_gaps} displays the corresponding gaps in percentage terms. The revised text reports: by 2016, the GDP-per-capita gap is 12.7\%, the total-GDP gap is 5.9\%, and the population gap is $-3.6$\%. Averaging over 2011--2022, the gaps are 6.8\% (GDP per capita), 5.0\% (total GDP), and $-3.6$\% (population).

\paragraph{Interpretation}
Yes, Canterbury's per-capita gap survives the decomposition. The positive post-quake GDP-per-capita divergence is not driven by denominator mechanics alone: total GDP itself rises above synthetic, while population remains below synthetic. Hence, the per-capita effect reflects the combined contribution of a stronger numerator and a smaller denominator relative to the counterfactual.

\paragraph{Comment 3.2 (placebo and timing inference).}
\textit{``Please provide stronger evidence that the Canterbury result is not driven by arbitrary treatment timing, and document the corresponding placebo diagnostics.''}

\paragraph{Response.}
Thank you for this suggestion. We have now added a systematic in-time placebo inference exercise for both countries and integrated it into the robustness discussion in the manuscript (Section~4).

We loop candidate treatment years across each case's pre-period (New~Zealand: 2000--2009; Chile: 1990--2009), re-estimate SCM for each candidate year, and compare placebo effects against the actual treatment-year estimate. The revised manuscript includes updated methods text and an expanded presentation of the placebo tests. The key conclusion is that the post-2011 Canterbury divergence is unique in magnitude relative to in-time placebos, while Maule's 2010 path remains within the placebo distribution (see Figure~\ref{fig:placebos}).

\paragraph{Comment 3.3 (SDID and bias-corrected SCM robustness).}
\textit{``Given the long post-treatment window, please implement SDID and a bias-corrected/penalized SCM (especially for Maule) to test robustness to inexact matching.''}

\paragraph{Response.}
Implemented. We added a new robustness module (\texttt{src/sdid\_bias\_corrected\_analysis.py}) that estimates:
\begin{enumerate}
  \item SDID for both Canterbury and Maule, and
  \item penalized (bias-corrected) SCM following the Abadie--L'Hour framework (reported for both countries; requested case Maule included).
\end{enumerate}

The revised manuscript reports these estimates in a new robustness table and figure (Table~\ref{tab:sdid_penalized}, Figure~\ref{fig:sdid_penalized_gaps}). Numerically:
\begin{itemize}
  \item \textbf{Canterbury}: baseline SCM average post-gap (2011--2019) is \(+7.8\%\); SDID is \(+7.8\%\); penalized SCM is \(+6.5\%\).
  \item \textbf{Maule}: baseline SCM average post-gap (2010--2019) is \(-1.7\%\); SDID is \(-5.1\%\); penalized SCM is \(-1.5\%\).
\end{itemize}

Hence, Canterbury's overshoot persists when optimizing both unit and time weights (SDID), while Maule remains non-positive under all alternative estimators. This directly addresses concerns about long-window robustness and edge-of-convex-hull matching for Maule.

\paragraph{Comment 3.6 (circularity and crowding-out in sectoral interpretation).}
\textit{``Sectoral interpretation appears circular; please quantify crowding-out and provide parallel Chile/New~Zealand sectoral inference.''}

\paragraph{Response.}
Implemented. We added a new reproducible sectoral module (\texttt{src/sectoral\_appendix\_analysis.py}) and integrated its outputs into the manuscript text and appendix.

\begin{itemize}
  \item We now include \textbf{parallel sectoral SCM figures for both countries}, not only Canterbury (new Maule figure added).
  \item We corrected quantitative wording: Canterbury construction rises from roughly \(6.1\%\) to \(10.0\%\), i.e., about \(+63\%\), \textbf{not} a doubling.
  \item We report placebo-based inference for sectoral outcomes in a new summary table.
  \item We quantify crowding-out in both \textbf{share} and \textbf{absolute level} terms.
\end{itemize}

Main quantitative additions (post-treatment averages):
\begin{itemize}
  \item \textbf{Canterbury}: construction share gap \(+2.53\) p.p. (pseudo-\(p=0.00\)); non-construction share gap \(-2.13\) p.p. (pseudo-\(p=0.00\)); non-construction \emph{level} gap \(+3.14\%\) of synthetic (pseudo-\(p=0.80\)).
  \item \textbf{Maule}: construction share gap \(+1.82\) p.p. (pseudo-\(p=0.17\)); non-construction share gap \(-2.01\) p.p. (pseudo-\(p=0.33\)); non-construction level gap \(-6.66\%\) of synthetic (pseudo-\(p=0.42\)).
\end{itemize}

Interpretation: Canterbury exhibits a strong \emph{share} reallocation toward construction, but no robust absolute contraction in non-construction output (consistent with temporary net stimulus plus reallocation). Maule does not exhibit a statistically robust sectoral overshoot.

We also added \textbf{alternative grouping sensitivity} for both countries (goods vs services non-construction aggregates), reported in the appendix table.

\section*{Reproducibility note}

All figures and tables cited above appear in the revised manuscript. The analysis code is available in the supplementary repository for verification.
